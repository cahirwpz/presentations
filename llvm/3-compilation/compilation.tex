\documentclass[8pt]{beamer}

\usepackage[utf8]{inputenc}
\usepackage[T1]{fontenc}
\usepackage{lmodern}
\usepackage{amsmath}
\usepackage[normalem]{ulem}
\usepackage{tikz}
\usepackage{hyperref}
\usepackage{listings}

\definecolor{links}{HTML}{2A1B81}
\hypersetup{colorlinks,linkcolor=,urlcolor=links}

\lstset{
  basicstyle=\footnotesize\ttfamily,
  numbers=left,
  numbersep=-2em,
  numberstyle=\color{gray}
}

\usetikzlibrary{shapes,arrows,automata}

\tikzset{
  vertex/.style={
    rectangle,
    rounded corners,
    draw=black, thick,
    text centered
  },
}

\mode<beamer>
{
  \usetheme{Frankfurt}
  \useoutertheme{miniframes}
  \setbeamercovered{transparent}
}

\subject{Talks}

\AtBeginSection[]
{
  \begin{frame}<beamer>{}
    \tableofcontents[currentsection]
  \end{frame}
}

\title[LLVM]{LLVM: Low Level Virtual Machine\\Compilation with LLVM framework}
\author[Krystian Bacławski]{\href{mailto:cahirwpz@cs.uni.wroc.pl}{Krystian Bacławski}}
\institute{Computer Science Department\\University of Wrocław}
\date{\today}

\begin{document}

\begin{frame}
\titlepage
\end{frame}

\section[Compilation]{Compilation with LLVM}
\subsection*{Compiler workflow}

\begin{frame}{Workflow}
  \begin{block}{Front-end stages}
    Completely up to the compiler implementation.
    \begin{itemize}
      \item Lexer (text $\rightarrow$ tokens)
      \item Parser (tokens $\rightarrow$ AST)
      \item Type checking (AST)
      \item Lowering (AST $\rightarrow$ IL)
      \item Transformations and optimizations (IL $\rightarrow$ IL)
      \item Repeat 4 and 5 to obtain IL suitable for code generation.
    \end{itemize}
  \end{block}
  
  \begin{block}{High-level stages}
    The compiler invokes LLVM APIs to obtain final LLVM IR.
    \begin{itemize}
      \item Lower IL to LLVM IR (IL $\rightarrow$ LLVM IR)
      \item SSA normalization (aka well-formedness).
      \item IR transformations (e.g. code instrumentation).
      \item Optimizations based on SSA form (target information can be
        employed).
    \end{itemize}
  \end{block}
\end{frame}

\begin{frame}{Workflow (cont.)}
  \begin{block}{Just-in-Time compilation}
    We can stop at this stage! Compile functions and their dependencies on
    demand. Create activation record and pass control to the function.
  \end{block}

  \begin{block}{Mid-level stages}
    Linking and link-time optimization. Still at binary code level. Little user
    control.
  \end{block}

  \begin{block}{Low-level stages}
    Fully controlled by LLVM target back-end. Optimizations at instruction
    level. Machine code generation:
    \begin{itemize}
      \item instruction selection
      \item instruction scheduling
      \item register assignment
      \item peephole optimization
    \end{itemize}
  \end{block}
\end{frame}

\subsection*{Where to begin?}

\begin{frame}{Choose your tools wisely!}
  \begin{alertblock}{}
    \begin{itemize}
      \item Programming language is only a tool!
      \item LLVM and Clang are written in \texttt{C++}
      \item If you choose not to use \texttt{C++}, keep in mind that you're
        still supposed to know it.
      \item Debugging can be tricky!
      \item Documentation for anything else than \texttt{C++} is scarce.
    \end{itemize}
  \end{alertblock}

  \begin{block}{Bindings for many languages (but experience may vary)}
    \begin{description}
      \item[\texttt{C}] only useful for building bindings to other languages
      \item[\texttt{Python}] convenient for simple tools
      \item[\texttt{Ocaml}] bindings included in LLVM source tree (but not
        of high quality)
      \item[\texttt{Haskell}] LLVM is used for GHC, so bindings are presumably
        well tested
    \end{description}
  \end{block}

  \begin{exampleblock}{}
    My personal choice? I went with \texttt{Ocaml} and I'm not pleased. Curious
    if \texttt{Haskell} is really better for that purpose. But that's
    completely different story.
  \end{exampleblock}
\end{frame}

\begin{frame}{LLVM's vocabulary}
  \begin{block}{Before you start reading tutorials, understanding following
      terms is crucial:}
    \begin{description}
      \item[Type] describes any type that exists within framework
      \item[Value] all (possibly named) first-class typed entities (incl.
        insns and funs)
      \item[Constant] data-like values may be immutable 
      \item[GlobalValue] variables and functions within specific module;
        may be externally visible; has \emph{Constant} address
      \item[GlobalVariable] may be initialized with \emph{Constant}; dynamic
        initialization requires use of \texttt{llvm.global\_ctors} and
        \texttt{llvm.global\_dtors}
      \item[Instruction] values that represent all available instructions
      \item[BasicBlock] contains instructions, must finish with terminator
        instruction; counterpart of node in \emph{CFG}s
      \item[Function] keeps track of a list of \emph{BasicBlock}s, formal
        \emph{Argument}s and a \emph{SymbolTable}
      \item[Argument] incoming formal argument to a \emph{Function}
      \item[Module] one or more constitute a program; merged by linker; keeps
        track of \emph{Function}s, \emph{GlobalVariable}s and \emph{SymbolTable}s
      \item[SymbolTable] provides a map that \emph{Function} or \emph{Module}
        uses for naming values
      \item[LLVMContext] provides isolation (separate compilation ctxs); stores
        type information
    \end{description}
  \end{block}
\end{frame}

\begin{frame}{Generating LLVM IR}
  \begin{exampleblock}{}
    Great tutorial for \texttt{C++} and \texttt{Ocaml} on LLVM's web page
    (Kaleidoscope). We can discuss an interesting excerpt from it right know,
    if you want.
  \end{exampleblock}

  \begin{block}{Kaleidoscope example covers:}
    \begin{itemize}
      \item external function declarations,
      \item function generation (prologue and epilogue),
      \item arithmetic operations on \emph{double}s,
      \item simple flow control (\texttt{if-then-else} and \texttt{for}
        statements)
      \item mutable and local variables,
      \item adding optimization passes,
      \item executing the code with JIT engine.
    \end{itemize}
  \end{block}
\end{frame}

\begin{frame}[fragile]{Tools}
  \begin{block}{Standard LLVM installation offers a set of tools}
    \begin{description}
      \item[llvm-as] assembler (\verb+.ll+ $\rightarrow$ \verb+.bc+)
      \item[llvm-dis] disassembler (\verb+.bc+ $\rightarrow$ \verb+.ll+)
      \item[opt] bitcode optimizer
      \item[llc] static compiler (\verb+.bc+ $\rightarrow$ \verb+.s+)
      \item[lli] directly execute programs from bitcode (uses JIT)
      \item[llvm-link] linker (\verb+.bc+$\{1,n\}$ $\rightarrow$ \verb+.bc+)
      \item[llvm-config] print LLVM compilation options
      \item[llvm-diff] structural \verb+diff+ over \verb+.ll+ files
      \item[llvm-stress] generate random \verb+.ll+ files (backend testing)
    \end{description}
  \end{block}
\end{frame}

\end{document}
