\documentclass[8pt]{beamer}

\usepackage[utf8]{inputenc}
\usepackage[T1]{fontenc}
\usepackage{lmodern}
\usepackage{amsmath}
\usepackage[normalem]{ulem}
\usepackage{tikz}
\usepackage{hyperref}
\usepackage{listings}

\definecolor{links}{HTML}{2A1B81}
\hypersetup{colorlinks,linkcolor=,urlcolor=links}

\lstset{
  basicstyle=\footnotesize\ttfamily,
  numbers=left,
  numbersep=-2em,
  numberstyle=\color{gray}
}

\usetikzlibrary{shapes,arrows,automata}

\tikzset{
  vertex/.style={
    rectangle,
    rounded corners,
    draw=black, thick,
    text centered
  },
}

\mode<beamer>
{
  \usetheme{Frankfurt}
  \useoutertheme{miniframes}
  \setbeamercovered{transparent}
}

\subject{Talks}

\AtBeginSection[]
{
  \begin{frame}<beamer>{}
    \tableofcontents[currentsection]
  \end{frame}
}

\title[LLVM]{LLVM: Low Level Virtual Machine\\An introduction to compiler infrastructure}
\author[Krystian Bacławski]{\href{mailto:cahirwpz@cs.uni.wroc.pl}{Krystian Bacławski}}
\institute{Computer Science Department\\University of Wrocław}
\date{\today}

\begin{document}

\begin{frame}
\titlepage
\end{frame}

\section[Introduction]{What is LLVM?}
\subsection*{Introduction}

\begin{frame}[fragile]{Introduction}
  \begin{block}{LLVM is a software framework for building compilers.}
    \begin{itemize}
      \item Well-defined intermediate code representation (aka LLVM IR -- \verb+.ll+ files).
      \item Binary representation of LLVM IR (\verb+.bc+ files).
      \item Code generators for many hardware architectures (\verb+x86+,
        \verb+mips+, \verb+arm+, \verb+sparc+, \verb+ppc+).
      \item Low-level optimizations (triggered during code generation phase).
      \item Just-in-Time compilation and execution subsystem.
      \item IR code transformations and optimizations (for use by an optimizing compiler).
      \item Linkable and executable file handling (\verb+ELF+, \verb+MachO+, \verb+COFF+).
      \item Tools like assembler, disassembler, optimizer and so on.
    \end{itemize}
  \end{block}

  \begin{block}{What LLVM is not?}
    \begin{description}
      \item[\sout{a compiler}] though one can use it to implement a mid-bottom part of a compiler.
      \item[\sout{a library}] set of libraries, tools and a run-time environment.
    \end{description}
  \end{block}
\end{frame}

\begin{frame}[fragile]{Applications}
  \begin{block}{Research platform:}
    \begin{itemize}
      \item Experiments with low-level and mid-level optimizations.
      \item Code profiling and instrumentation.
      \item ISA (instruction set architecture) design.
      \item Compilers for domain specific languages.
    \end{itemize}
  \end{block}

  \begin{exampleblock}{Notable LLVM uses:}
    \begin{itemize}
      \item Clang (C, C++, ObjC and ObjC++ front-end)
      \item Compilers: Haskell, NVIDIA CUDA, Erlang HiPE,
      \item Interpreters: PyPy, Mono, IcedTea, ClamAV,
      \item GPGPU: OpenCL, Apple's OpenGL,
      \item LLDB -- GNU Debugger competitor,
      \item ASAN and TSAN -- code instrumentation for debugging purposes,
      \item and so on\ldots (\href{http://llvm.org/Users.html}{LLVM Users})
    \end{itemize}
  \end{exampleblock}
\end{frame}

\begin{frame}[fragile]{Competition}
  \begin{block}{LLVM together with Clang and LLDB tries to compete with GNU Toolchain}
    \begin{itemize}
      \item object and archive manipulation (\verb+ar+, \verb+ranlib+,
        \verb+nm+, \verb+objdump+, \verb+readobj+)
      \item profiler (\verb+prof+, \verb+cov+)
      \item compiler, assembler and linker (\verb+clang+)
      \item debugger (\verb+lldb+)
    \end{itemize}
  \end{block}

  \begin{alertblock}{Especially Clang is worth a look!}
    \begin{itemize}
      \item It offers easy to understand diagnostic messages:
        \begin{itemize}
          \item especially for C++ templates and namespaces,
          \item more meaningful warnings!
        \end{itemize}
      \item \ldots superior (memory and time) efficiency compared to GCC,
      \item \ldots and a fast C / C++ parser library for:
        \begin{itemize}
          \item high-level optimizations,
          \item static analysis,
          \item program transformations.
        \end{itemize}
      \item Is an area of interesting (and practical) research:
        \begin{itemize}
          \item modules for C language,
          \item thread-safety annotation checking.
        \end{itemize}
    \end{itemize}
  \end{alertblock}
\end{frame}

\end{document}
