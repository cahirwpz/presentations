\documentclass[8pt]{beamer}

\usepackage[utf8]{inputenc}
\usepackage[T1]{fontenc}
\usepackage{lmodern}
\usepackage{amsmath}
\usepackage[normalem]{ulem}
\usepackage{tikz}
\usepackage{hyperref}
\usepackage{listings}

\definecolor{links}{HTML}{2A1B81}
\hypersetup{colorlinks,linkcolor=,urlcolor=links}

\lstset{
  basicstyle=\footnotesize\ttfamily,
  numbers=left,
  numbersep=-2em,
  numberstyle=\color{gray}
}

\usetikzlibrary{shapes,arrows,automata}

\tikzset{
  vertex/.style={
    rectangle,
    rounded corners,
    draw=black, thick,
    text centered
  },
}

\mode<beamer>
{
  \usetheme{Frankfurt}
  \useoutertheme{miniframes}
  \setbeamercovered{transparent}
}

\subject{Talks}

\AtBeginSection[]
{
  \begin{frame}<beamer>{}
    \tableofcontents[currentsection]
  \end{frame}
}

\title[LLVM]{LLVM: Low Level Virtual Machine\\Exception handling}
\author[Krystian Bacławski]{\href{mailto:cahirwpz@cs.uni.wroc.pl}{Krystian Bacławski}}
\institute{Computer Science Department\\University of Wrocław}
\date{\today}

\begin{document}

\begin{frame}
\titlepage
\end{frame}

\section[Exception handling]{Exception handling with LLVM}
\subsection*{}

\begin{frame}[fragile]{Constituents of exception handling}
  \begin{block}{Two ways to implement exceptions:}
    \begin{description}
      \item[non-local jumps] known as \texttt{setjmp} and \texttt{longjmp};
        saves whole context at \texttt{setjmp} thus expensive; no value
        associated with \texttt{throw} statement; how to clean up the stack at
        \texttt{longjmp}?
      \item[stack unwind] requires elaborate runtime and compiler support;
        during stack unwind, which registers were modified
        (\texttt{DWARF2}-like information) and how to restore them? exception
        conveys certain typed value; for each function a static table of
        handler routines;
    \end{description}
  \end{block}

  \begin{alertblock}{}
    Although LLVM delivers quite generic exception handling, the slides are
    \verb|C++| oriented.
  \end{alertblock}

  \begin{exampleblock}{References}
    \begin{itemize}
      \item
        \href{http://llvm.org/devmtg/2011-09-16/EuroLLVM2011-ExceptionHandling.pdf}{The
          new LLVM exception handling scheme}
      \item \href{http://llvm.org/docs/ExceptionHandling.html}{Exception Handling
          in LLVM}
      \item \href{http://mentorembedded.github.com/cxx-abi/abi-eh.html}{Itanium
          C++ ABI: Exception Handling}
    \end{itemize}
  \end{exampleblock}
\end{frame}

\begin{frame}[fragile]{Implementation of \texttt{throw}, \texttt{try},
    \texttt{catch}, \texttt{finally}.}
  \begin{block}{How to throw an exception?}
    \begin{itemize}
      \item Allocate space for an exception structure
        (\verb+__cxa_allocate_exception+) outside of the stack, and fill it in.
      \item Call runtime function with the structure as an argument
        (\verb+__cxa_throw+) that will initiate stack unwinding.
    \end{itemize}
  \end{block}

  \begin{block}{How to construct \texttt{try} statement?}
    We need to decorate each function call within \texttt{try} statement. Thus
    requirement for \textbf{invoke} instruction that will behave as a function
    call instruction, but will specify \textbf{landing pad} in case stack
    unwinding was triggered.
  \end{block}

  \begin{block}{What the heck is personality function?}
    For each stack frame, \textbf{personality function} answers a question: does
    a function want to handle the exception or not (information stored in
    \verb+.ex_frame+). It's invoked by stack unwinder. If the answer was yes,
    then control flow is passed to a \textbf{landing pad} together with the
    exception structure.
  \end{block}
\end{frame}

\begin{frame}[fragile]{Implementation of \texttt{throw}, \texttt{try},
    \texttt{catch}, \texttt{finally} (cont.)}
  \begin{block}{Exception filtering}
    Some languages allow to specify that a function is not supposed to throw
    certain exceptions. If it happens, a handler should make a call
    to \verb+__cxa_call_unexpected+.
  \end{block}

  \begin{block}{Landing pad}
    For \verb|C++|, the \verb+landingpad+ instruction returns a pair of the pointer to
    the exception structure and the selector value. If the selector is:
    \begin{description}
      \item[positive] it means that an exception, which type corresponds to
        selector value (\verb+llvm.eh.typeid.for+ intrinsic), was caught,
      \item[zero] cleanup action should be performed (\textbf{finally} clause)
        and stack unwinding should be resumed by \verb+resume+ instruction,
      \item[negative] unexpected exception has been received (\textbf{filter} clause).
    \end{description}
  \end{block}
\end{frame}

\begin{frame}[fragile]{LLVM IR}
  \begin{exampleblock}{Support for exception handling.}
    \begin{lstlisting}[basicstyle=\tiny\ttfamily,numbers=none]
      ; Invoke Test function. If executed normally go to %Continuation,
      ; otherwise stack unwinding goes to %TestCleanup landing pad.
      %retval = invoke i32 @Test(i32 15) to label %Contination
                                         unwind label %TestCleanup

      ; A landing pad which can catch an integer.
      %res = landingpad { i8*, i32 } personality i32 (...)* @__gxx_personality_v0
                                     catch i8** @_ZTIi
      ; A landing pad that is a cleanup.
      %res = landingpad { i8*, i32 } personality i32 (...)* @__gxx_personality_v0
                                     cleanup
      ; A landing pad that catches everything.
      %res = landingpad { i8*, i32 } personality i32 (...)* @__gxx_personality_v0
                                     catch null
      ; A landing pad which can catch an integer and can only throw a double.
      %res = landingpad { i8*, i32 } personality i32 (...)* @__gxx_personality_v0
                                     catch i8** @_ZTIi
                                     filter [1 x i8**] [@_ZTId]

      ; Obtains a unique identifier of a type described with _ZTIi structure (an integer).
      %sel = tail call i32 @llvm.eh.typeid.for(i8* bitcast (i8** @_ZTIi to i8*)) nounwind

      ; Within exception handling code: continue stack unwinding with given value.
      resume { i8*, i32 } %exn
    \end{lstlisting}
  \end{exampleblock}
\end{frame}

\begin{frame}[fragile]{EH example: code in \texttt{C++}}
  \begin{exampleblock}{Let's try to translate following \texttt{C++} code into
      \texttt{LLVM IR}:}
    \begin{lstlisting}
      extern void bar();

      void foo() throw (const char *) {
        try {
          bar();
        } catch (int) {
          ...
        }
      }
    \end{lstlisting}
  \end{exampleblock}
\end{frame}

\begin{frame}[fragile]{EH example: translation to \texttt{LLVM IR}}
  \begin{lstlisting}[language=C,basicstyle=\tiny\ttfamily]
    @_ZTIPKc = external constant i8*
    @_ZTIi = external constant i8*

    define void @_Z3foov() uwtable ssp {
    entry:
      invoke void @_Z3barv() to label %try.cont unwind label %lpad

    lpad:
      %0 = landingpad { i8*, i32 } personality i8* bitcast (i32 (...)* @__gxx_personality_v0 to i8*)
                                   catch i8* bitcast (i8** @_ZTIi to i8*)
                                   filter [1 x i8*] [i8* bitcast (i8** @_ZTIPKc to i8*)]
      %1 = extractvalue { i8*, i32 } %0, 0
      %2 = extractvalue { i8*, i32 } %0, 1
      %3 = tail call i32 @llvm.eh.typeid.for(i8* bitcast (i8** @_ZTIi to i8*)) nounwind
      %matches = icmp eq i32 %2, %3
      br i1 %matches, label %catch, label %filter.dispatch

    filter.dispatch:
      %ehspec.fails = icmp slt i32 %2, 0
      br i1 %ehspec.fails, label %ehspec.unexpected, label %eh.resume

    ehspec.unexpected:
      tail call void @__cxa_call_unexpected(i8* %1) no return
      unreachable

    catch:
      %4 = tail call i8* @__cxa_begin_catch(i8* %1) nounwind
      tail call void @__cxa_end_catch() nounwind
      br label %try.cont

    try.cont:
      ret void

    eh.resume:
      resume { i8*, i32 } %0
    }
  \end{lstlisting}
\end{frame}

\end{document}
