\PassOptionsToPackage{subsection=false}{beamerouterthememiniframes}
\documentclass[8pt]{beamer}

\usepackage[utf8]{inputenc}
\usepackage[T1]{fontenc}
\usepackage{lmodern}
\usepackage{amsmath}
\usepackage[normalem]{ulem}
\usepackage{tikz}
\usepackage{hyperref}
\usepackage{listings}
\usepackage{color}
\usepackage{fancyvrb}
\usepackage{minted}

\usepackage{mathptmx}
\usepackage[scaled=0.9]{helvet}

\definecolor{links}{HTML}{2A1B81}
\hypersetup{colorlinks,linkcolor=,urlcolor=links}

\usetikzlibrary{shapes,arrows,automata}

\tikzset{
  vertex/.style={
    rectangle,
    rounded corners,
    draw=black, thick,
    text centered
  },
}

\mode<beamer>
{
  \usetheme{Frankfurt}
  \useoutertheme{miniframes}
  \setbeamercovered{transparent}
}

\subject{Talks}

\AtBeginSection[]
{
  \begin{frame}<beamer>{}
    \tableofcontents[currentsection]
  \end{frame}
}

\title[LLVM]{LLVM: Low Level Virtual Machine\\LLVM Intermediate Representation}
\author[Krystian Bacławski]{\href{mailto:cahirwpz@cs.uni.wroc.pl}{Krystian Bacławski}}
\institute{Computer Science Department\\University of Wrocław}
\date{\today}

\begin{document}

\begin{frame}
\titlepage
\end{frame}

\section[Language]{LLVM language}

\subsection*{LLVM IR basics}

\begin{frame}[fragile]{LLVM machine model}
  \begin{block}{}
    \begin{itemize}
      \item Register based machine -- infinite number of virtual registers.
      \item RISC-like load-store approach to memory access.
      \item Implicit stack -- no stack management else than variable placement.
      \item Normalized instruction stream in SSA form (implicit CFG).
      \item Strongly typed:
        \begin{itemize}
          \item explicit coercions between numeric machine types,
          \item typed pointers!
        \end{itemize}
      \item Heavily influenced by C language but versatile enough.
    \end{itemize}
  \end{block}

  \begin{block}{Some features}
    \begin{itemize}
      \item Support for accurate garbage collection.
      \item Support for Just-in-Time compilation.
      \item Language agnostic exception handling (incl. DWARF2 stack unwinding).
      \item Memory model for concurrent programming.
      \item Branch profiling.
    \end{itemize}
  \end{block}
\end{frame}

\begin{frame}[fragile]{Type system}
  \begin{block}{Types}
    \begin{itemize}
      \item C-like types: integer, floating point, void, array, function,
        pointer,
      \item label -- basic block entry,
      \item structure -- fields are enumerated, but not named,
      \item vector -- arbitrary length, for SIMD processors,
      \item opaque -- forward declaration,
      \item metadata -- code annotation.
    \end{itemize}
    \textbf{More information here: \href{http://llvm.org/docs/LangRef.html\#type-system}{Type System}}
  \end{block}

  \begin{alertblock}{}
    Most types are first-class! It's not allowed to pass around a function type
    -- but a pointer to function type is all right.
  \end{alertblock}

  \begin{block}{Type notation}
    \begin{description}
      \item[i1] boolean represented as a single-bit integer
      \item[array] \verb+[ <#elements> x <element type> ]+
      \item[vector] \verb+< <#elements> x <element type> >+
      \item[structure] \verb+type { <type list> }+
    \end{description}
  \end{block}
\end{frame}

\begin{frame}[fragile]{Values}
  \begin{block}{Special values}
    \begin{description}
      \item[null] invalid pointer
      \item[boolean] \textbf{true} and \textbf{false}
      \item[undef] unspecified bit-pattern value, may represent poison value
      \item[zeroinitializer] for initialization of aggregate values
      \item[BB's address] for \textbf{indirectbr} instruction
    \end{description}
    \textbf{More information here:
      \href{http://llvm.org/docs/LangRef.html\#constants}{Constants}}
  \end{block}

  \begin{block}{Metadata}
    Structure-like node that convey extra information about the code. There
    are metadata nodes recognized by the optimisers and code generator.
    \begin{minted}{llvm}
      !0 = metadata !{ i32 0, metadata !"test" }
      call void @llvm.dbg.value(metadata !24, i64 0, metadata !25)
      %indvar.next = add i64 %indvar, 1, !dbg !21
    \end{minted}
    \textbf{More information here:
      \href{http://llvm.org/docs/LangRef.html\#metadata-nodes-and-metadata-strings}{Metadata
        nodes and metadata-strings}.}
  \end{block}
\end{frame}

\begin{frame}[fragile]{Instructions}
  \begin{block}{}
    \begin{itemize}
      \item Arithmetic instructions:
        \begin{itemize}
          \item integer: \verb+add+, \verb+sub+, \verb+mul+, \verb+udiv+,
            \verb+sdiv+, \verb+urem+, \verb+srem+, \verb+icmp+
          \item floating-point: \verb+fadd+, \verb+fsub+, \verb+fmul+, \verb+fdiv+,
            \verb+frem+, \verb+fcmp+
        \end{itemize}

      \item Bitwise instructions:
        \begin{itemize}
          \item bit-shift: \verb+shl+, \verb+lshr+, \verb+ashr+
          \item boolean: \verb+and+, \verb+or+, \verb+xor+
        \end{itemize}

      \item Control flow instructions:
        \begin{itemize}
          \item function calls: \verb+invoke+, \verb+call+, \verb+ret+
          \item jumping: \verb+br+, \verb+switch+, \verb+indirectbr+
          \item other: \verb+phi+, \verb+select+
        \end{itemize}

      \item Memory access instructions:
        \begin{itemize}
          \item stack space allocation: \verb+alloca+
          \item generic accessors: \verb+load+, \verb+store+
        \end{itemize}

      \item Compound data access instructions:
        \begin{itemize}
          \item vectors: \verb+extractelement+, \verb+insertelement+, \verb+shufflevector+
          \item aggregates: \verb+extractvalue+, \verb+insertvalue+
          \item other: \verb+getelementptr+
        \end{itemize}

      \item Data conversion instructions:
        \begin{itemize}
          \item integer: \verb+trunc+, \verb+zext+ \verb+sext+
          \item floating-point: \verb+fptrunc+, \verb+fpext+, \verb+fptoui+,
            \verb+fptosi+, \verb+uitofp+, \verb+sitofp+,
          \item pointers: \verb+ptrtoint+, \verb+inttoptr+, \verb+bitcast+
        \end{itemize}
    \end{itemize}
    \textbf{More information here: \href{http://llvm.org/docs/LangRef.html\#instruction-reference}{Instruction reference}}
  \end{block}
\end{frame}

\begin{frame}[fragile]{Non-control flow instruction examples}
  \begin{exampleblock}{}
    \begin{minted}[gobble=6,linenos,fontsize=\small]{llvm}
      %r0 = add i32 4, %num                             ; r0 := 4 + %num
      %r1 = fdiv float 11.0, %fp                        ; r1 := %fp / 11.0
      %r2 = or i1 %b1, %b2                              ; r2 := %b1 || %b2
      %r4 = extractelement <4 x i32> %vec, i32 0        ; r4 := %vec[0]
      %r5 = insertvalue {i32, float} %agg, float %v, 1  ; r5 := %agg{[1] = %v}
      %ptr = alloca i32                                 ; {i32 *}
      store i32 3, i32* %ptr                            ; void : *(%ptr) = 3
      %v = load i32* %ptr                               ; i32 : v := *(%ptr)
      %p = icmp eq i32 %n, 0                            ; bool : p := (%n = 0)
      %z = select i1 %p, i8 %x, i8 %y                   ; %z = %p ? %x : %y
    \end{minted}
  \end{exampleblock}

  \begin{block}{Comments:}
    \begin{itemize}
      \item Integer: different signedness and word size possible, overflow
        control.
      \item Floating-point: different word size possible, compliance with
        \verb+IEEE754+.
      \item Boolean calculation for \verb+i1+ type.
      \item \verb+alloca+ is the only instruction dealing with stack.
      \item SSA is performed only for registers, not memory locations.
      \item Choice of type essential for optimization.
    \end{itemize}
  \end{block}
\end{frame}

\begin{frame}[fragile]{Accessing aggregates}
  \begin{exampleblock}{}
    \begin{minted}[gobble=6,linenos,fontsize=\small]{c}
      struct RT { /* Structure with complex types */
        char A;
        int B[10][20];
        char C;
      };

      struct ST {
        int X;
        double Y;
        struct RT Z; /* ST contains an instance of RT embedded in it */
      };

      int *foo(struct ST *s) {
        return &s[1].Z.B[5][13];
      }
    \end{minted}
  \end{exampleblock}
  
  \begin{exampleblock}{}
    \begin{minted}[gobble=6,linenos,fontsize=\small]{llvm}
      %struct.ST = type { i32, double, %struct.RT }
      %struct.RT = type { i8, [10 x [20 x i32]], i8 }

      define i32* @foo(%struct.ST* %s) {
        %1 = getelementptr %struct.ST* %s, i64 1, i32 2, i32 1, i64 5, i64 13
        ret i32* %1
      }
    \end{minted}
  \end{exampleblock}
\end{frame}

\begin{frame}[fragile]{Control flow instruction examples}
  \begin{exampleblock}{Infinite loop that counts from 0 on up\ldots}
    \begin{minted}[gobble=6,linenos,fontsize=\small]{llvm}
      LoopHeader:
        ...
        br label %Loop

      Loop:
        %IndVar = phi i32 [ 0, %LoopHeader ], [ %NextIndVar, %Loop ]
        %NextIndVar = add i32 %IndVar, 1
        br label %Loop
    \end{minted}
  \end{exampleblock}

  \begin{block}{First basic control flow graph!}
    \begin{itemize}
      \item $\phi(\ldots)$ syntax: \verb+<result> = phi <type> [<val0>, <label0>], ...+
      \item $\phi(\ldots)$ states:
        \begin{itemize}
          \item how control flow can reach given node?
          \item which value to select depending on previous node?
        \end{itemize}
    \end{itemize}
  \end{block}
\end{frame}

\begin{frame}[fragile]{Call instruction examples}
  \begin{exampleblock}{Function call examples:}
    \begin{minted}[gobble=6,linenos,fontsize=\small]{llvm}
      %retval = call i32 @test(i32 %argc)
      call i32 (i8*, ...)* @printf(i8* %msg, i32 12, i8 42)
      %X = tail call i32 @foo()
      %struct.A = type { i32, i8 }
      %r = call %struct.A @bar()
      %Z = call void @foo() noreturn
    \end{minted}
  \end{exampleblock}

  \begin{block}{Comments:}
    \begin{itemize}
      \item Different calling convention (\verb+fastcc+, \verb+ghc+, \ldots)
      \item \verb+varargs+ functions supported.
      \item Can return compound and aggregate types.
      \item Special attribute for tail-call optimization.
      \item \verb+invoke+ supports exceptions.
    \end{itemize}
  \end{block}
\end{frame}

\begin{frame}[fragile]{Compilation unit}
  \begin{block}{Each compilation unit comprise of:}
    \begin{itemize}
      \item locally visible compound type declarations,
      \item function definitions (with attributes / visibility specifiers),
      \item global variables,
      \item external function (pointer) type declarations,
      \item architecture information and metadata (e.g. for debugging)
    \end{itemize}
  \end{block}

  \begin{exampleblock}{All four composites of LLVM module:}
    \begin{minted}[gobble=6,linenos,fontsize=\small]{llvm}
      target datalayout = "..."

      @.str = private unnamed_addr constant [13 x i8] c"hello world\0A\00"

      declare i32 @puts(i8* nocapture) nounwind

      define i32 @main() 
        %hello = getelementptr [13 x i8]* @.str, i64 0, i64 0
        call i32 @puts(i8* %hello)
        ret i32 0
      }

      !1 = metadata !{i32 42}
    \end{minted}
  \end{exampleblock}
\end{frame}

\section[Examples]{LLVM IR Examples}
\subsection*{LLVM IR examples}

\begin{frame}[fragile]{Add two integers together}
  \begin{exampleblock}{}
    \begin{minted}[gobble=6,linenos,fontsize=\footnotesize]{c}
      unsigned add1(unsigned a, unsigned b) {
        return a+b;
      }

      unsigned add2(unsigned a, unsigned b) {
        if (a == 0) return b;
        return add2(a-1, b+1);
      }
    \end{minted}
  \end{exampleblock}

  \begin{exampleblock}{}
    \begin{minted}[gobble=6,linenos,fontsize=\footnotesize]{llvm}
      define i32 @add1(i32 %a, i32 %b) {
        entry:
          %tmp1 = add i32 %a, %b
          ret i32 %tmp1
      }

      define i32 @add2(i32 %a, i32 %b) {
        entry:
          %tmp1 = icmp eq i32 %a, 0
          br i1 %tmp1, label %done, label %recurse

        recurse:
          %tmp2 = sub i32 %a, 1
          %tmp3 = add i32 %b, 1
          %tmp4 = call i32 @add2(i32 %tmp2, i32 %tmp3)
          ret i32 %tmp4

        done:
          ret i32 %b
      }
    \end{minted}
  \end{exampleblock}
\end{frame}

\begin{frame}[fragile]{How to use LLVM tools?}
  \begin{block}{Generate some code from C source}
    \begin{minted}[gobble=6,linenos]{bash}
      clang -S -emit-llvm -o test.ll test.c
      clang -S -O2 -emit-llvm -o test.ll test.c
      llc -o test.s test.ll
      llc -march=ppc32 -o test.s test.ll
    \end{minted}
  \end{block}

  \begin{block}{View Control Flow Graph}
    \begin{minted}[gobble=6,linenos]{bash}
      opt -view-cfg -S test.ll
      opt -mem2reg -view-cfg -S test.ll
    \end{minted}
  \end{block}
\end{frame}

\section*{The End}

\begin{frame}[fragile]{LLVM links}
  \begin{block}{Research links:}
    \begin{enumerate}
      \item \href{http://llvm.org/pubs/}{LLVM Related Publications}
      \item \href{http://llvm.org/devmtg/}{LLVM Developers' Meeting}
    \end{enumerate}
  \end{block}

  \begin{block}{Presentation and papers about LLVM and verification:}
    \begin{enumerate}
      \item \href{http://llvm.org/devmtg/2013-11/#talk5}
         {Verifying optimizations using SMT solvers}
      \item \href{http://www.cs.rutgers.edu/~santosh.nagarakatte/pldi2013.pdf}
        {Formal Verification of SSA-Based Optimizations for LLVM}
      \item \href{http://llvm.org/devmtg/2012-11/#talk5}
        {Verified LLVM: Formalizing the semantics of the LLVM Intermediate
          Representation for Verified Program Transformations}
    \end{enumerate}
  \end{block}
\end{frame}

\begin{frame}{}
  \vspace*{\stretch{2}}
  \begin{center}
    {\Huge Questions?}
  \end{center}
  \vspace{\stretch{3}}
\end{frame}

\end{document}
