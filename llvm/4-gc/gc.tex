\documentclass[8pt]{beamer}

\usepackage[utf8]{inputenc}
\usepackage[T1]{fontenc}
\usepackage{lmodern}
\usepackage{amsmath}
\usepackage[normalem]{ulem}
\usepackage{tikz}
\usepackage{hyperref}
\usepackage{listings}

\definecolor{links}{HTML}{2A1B81}
\hypersetup{colorlinks,linkcolor=,urlcolor=links}

\lstset{
  basicstyle=\footnotesize\ttfamily,
  numbers=left,
  numbersep=-2em,
  numberstyle=\color{gray}
}

\usetikzlibrary{shapes,arrows,automata}

\tikzset{
  vertex/.style={
    rectangle,
    rounded corners,
    draw=black, thick,
    text centered
  },
}

\mode<beamer>
{
  \usetheme{Frankfurt}
  \useoutertheme{miniframes}
  \setbeamercovered{transparent}
}

\subject{Talks}

\AtBeginSection[]
{
  \begin{frame}<beamer>{}
    \tableofcontents[currentsection]
  \end{frame}
}

\title[LLVM]{LLVM: Low Level Virtual Machine\\An introduction to compiler infrastructure}
\author[Krystian Bacławski]{\href{mailto:cahirwpz@cs.uni.wroc.pl}{Krystian Bacławski}}
\institute{Computer Science Department\\University of Wrocław}
\date{\today}

\begin{document}

\begin{frame}
\titlepage
\end{frame}

\section[Garbage collection]{Garbage collection with LLVM}
\subsection*{}

\begin{frame}{Introduction}
  \begin{alertblock}{}
    Runtime generates a directed graph of memory blocks -- pointers are
    outgoing edges. Some blocks are in use (active). How are we supposed to
    find inactive blocks?
  \end{alertblock}

  \begin{block}{\textit{GC roots}}
    They point to a digraph of active memory blocks. Exist in three forms:
    \begin{itemize}
      \item \textbf{on stack},
      \item \textbf{in registers},
      \item \textbf{as global variables}.
    \end{itemize}
  \end{block}

  \begin{block}{Two approaches to garbage collection}
    \begin{description}
      \item[conservative] Traverses the stack and registers to find gc roots.
        Interprets data as pointers. False positives possible. Inhibits object
        moving. Can be added to language without GC. Good for non-type-safe
        languages. 
      \item[accurate] Maintains stack map. Requires compiler support.
        Compaction possible (but requires stack map update). Some optimization
        techniques less effective.
    \end{description}
  \end{block}
\end{frame}

\begin{frame}[fragile]{Interfacing with garbage collector}
  \begin{block}{LLVM delivers mechanisms for accurate garbage collection:}
    \begin{itemize}
      \item creation of points, suitable for garbage collection, within the
        code, 
      \item computation of the stack map (gc roots on stack),
      \item emission of read / write barriers (concurrent garbage collection).
    \end{itemize}
  \end{block}

  \begin{block}{How to incorporate GC into our compiler?}
    \begin{enumerate}
      \item Write a runtime library or find an existing one which implements a
        GC heap.
      \item Emit code that can interface with garbage collector.
      \item Write a compiler plugin to interface LLVM with the runtime library.
      \item Load the plugin into the compiler.
      \item Link program executables with the runtime.
    \end{enumerate}
  \end{block}

  \begin{exampleblock}{}
    Lucikly enough, LLVM delivers built-in \textbf{ShadowStack} code generator
    -- i.e. a code transformation plugin that instruments the code to work with
    the GC -- and minimal run-time support.
  \end{exampleblock} 
\end{frame}

\begin{frame}[fragile]{How to get it working with \textit{ShadowStack}?}
  \begin{block}{Runtime:}
    \begin{itemize}
      \item find a runtime library which implements a GC heap (e.g. Boehm GC),
      \item implement a registry for global roots,
      \item design a binary interface for type maps (heap references),
      \item implement a collection routine.
    \end{itemize}
  \end{block}

  \begin{block}{Code emit:}
    \begin{itemize}
      \item initialize GC in the main function,
      \item use the \verb+gc "..."+ attribute to enable GC code generation,
      \item use \verb+@llvm.gcroot+ to mark stack roots,
      \item use \verb+@llvm.gcread+ / \verb+@llvm.gcwrite+ to manipulate GC
        references,
      \item allocate memory using the GC allocation,
      \item generate type maps according to your runtime's binary interface,
    \end{itemize}
  \end{block}

  \begin{block}{Interface with run-time:}
    \begin{itemize}
      \item compile LLVM's stack map to the binary form expected by the
        runtime.
    \end{itemize}
  \end{block}
\end{frame}

\begin{frame}[fragile]{GC roots handling}
  \begin{block}{Mark variables on stack to be tracked for garbage collection with:}
    \begin{lstlisting}[language=C,basicstyle=\ttfamily,numbers=none]
      void @llvm.gcroot(i8** %ptrloc, i8* %metadata)
    \end{lstlisting}
    \begin{description}
      \item[\texttt{ptrloc}] must be a value referring to a pointer on stack
        (\texttt{alloca} instruction)
      \item[\texttt{metadata}] a pointer to data associated with
        \texttt{ptrloc}, must be a constant or global value address
    \end{description}
  \end{block}

  \begin{block}{Can we avoid requiring heap objects to have \emph{isa} pointers
      or tag bits?}
    If \texttt{metadata} is specified, its value will be tracked along with the
    location of the pointer in the stack frame.
  \end{block}

  \begin{alertblock}{Important notes:}
    \begin{enumerate}
      \item All calls to \verb|llvm.gcroot| must reside inside the first basic
        block.
      \item All intermediate values must be marked with \verb|llvm.gcroot|.
    \end{enumerate}
  \end{alertblock}
\end{frame}

\begin{frame}[fragile]{Built-in stack crawler routine}
  \begin{exampleblock}{}
    \begin{lstlisting}[language=C,basicstyle=\tiny\ttfamily]
      /* The map for a single function's stack frame.  One of these is compiled
         as constant data into the executable for each function. */
      struct FrameMap {
        int32_t NumRoots;    //< Number of roots in stack frame.
        int32_t NumMeta;     //< Number of metadata entries.  May be < NumRoots.
        const void *Meta[0]; //< Metadata for each root.
      };

      /* Each stack frame has one of these. */
      struct StackEntry {
        StackEntry *Next;    //< Link to next stack entry (the caller's).
        const FrameMap *Map; //< Pointer to constant FrameMap.
        void *Roots[0];      //< Stack roots (in-place array).
      };

      /* Functions push and pop onto this in their prologue and epilogue. */
      StackEntry *llvm_gc_root_chain;

      /* Visitor is a function to be invoked for every GC root on the stack. */
      void visitGCRoots(void (*Visitor)(void **Root, const void *Meta)) {
        for (StackEntry *R = llvm_gc_root_chain; R; R = R->Next) {
          unsigned i = 0;

          /* For roots [0, NumMeta), the metadata pointer is in the FrameMap. */
          for (unsigned e = R->Map->NumMeta; i != e; ++i)
            Visitor(&R->Roots[i], R->Map->Meta[i]);

          /* For roots [NumMeta, NumRoots), the metadata pointer is null. */
          for (unsigned e = R->Map->NumRoots; i != e; ++i)
            Visitor(&R->Roots[i], NULL);
        }
      }
    \end{lstlisting}
  \end{exampleblock}
\end{frame}

\end{document}
